\documentclass[a4paper,10pt]{article}

\usepackage{fontspec} 					%for loading fonts
\usepackage{xunicode,xltxtra,url,parskip} 	%other packages for formatting
\RequirePackage{color,graphicx}
\usepackage{fontawesome}
\usepackage{hyperref}
\definecolor{linkcolour}{rgb}{0,0.2,0.6}
\hypersetup{colorlinks,breaklinks,urlcolor=linkcolour, linkcolor=linkcolour}

\usepackage{enumitem}
\setlist{nolistsep}

\usepackage{geometry}
\geometry{
    a4paper,
    % total={170mm,257mm},
    left=25mm,
    right=25mm,
    top=15mm,
    bottom=15mm,
}

%CV Sections inspired by:
%http://stefano.italians.nl/archives/26
\usepackage{titlesec}
\titleformat{\section}{\bfseries\scshape\raggedright}{}{0em}{}[\titlerule]
\titlespacing{\section}{0pt}{2pt}{2pt}

\begin{document}


\pagestyle{empty} % non-numbered pages

\par{
    \center
{\Huge SHIH-MING WANG}\\
\faEnvelope: \href{mailto:swang150@ucsc.edu}{swang150@ucsc.edu} \\
\faUser: \href{http://ipod825.github.io/}{http://ipod825.github.io/}\\
\faLinkedin: \href{https://www.linkedin.com/in/shih-ming-wang-73aa3769/}{https://www.linkedin.com/in/shih-ming-wang}\\
\bigskip\par}

%Section: Education
\section{Education}
    \textbf{Ph.D., Computer Science}, University of California, Santa Cruz \ (GPA:3.96/4.0)\hfill Sep. 2016-Present\\
    \textbf{M.S., Computer Science}, National Taiwan University \ (GPA:4.21/4.3)\hfill Sep. 2012 - June 2014 \\
    \textbf{B.S., Electrical Engineering}, National Taiwan University \hfill Sep. 2008 - June 2012

\section{Publication}
\textbf{Wang, S. M.}, \& Ku, L. W. (2006). ANTUSD: A Large Chinese Sentiment Dictionary.

\textbf{Wang, S. M.}, Tung, Y. F., \& Yu, T. L. (2014, July). Investigation on efficiency of optimal mixing on various linkage sets. In 2014 IEEE Congress on Evolutionary Computation (CEC) (pp. 2475-2482). IEEE.

\textbf{Wang, S. M.}, Wu, J. W., Chen, W. M., \& Yu, T. L. (2013, July). Design of test problems for discrete estimation of distribution algorithms. In Proceedings of the 15th annual conference on Genetic and evolutionary computation (pp. 407-414). ACM.

\section{Research Experience}
\textbf{Machine Learning Research Assistant, Academia Sinica}, Taipei, Taiwan \hfill Aug. 2015 - July 2016 \par
\begin{itemize}
    \item \textbf{Lightweight Discourse CNN Model for Sentiment Analysis}
    \begin{itemize}
        \item Proposed a new CNN model incorporating the knowledge of discourse rules (e.g contrast and concession) and the learning ability of the deep neural network.
        \item Performed experiments on several well-known sentiment analysis datasets (e.g. Stanford Sentiment Tree-bank) showing the simplicity and effectiveness of the proposed model.
    \end{itemize}
    \item \textbf{Sensing Emotions in Text Messages}
    \begin{itemize}
        \item Supervised on the project aiming to build a system that automatically conveys the emotion of received text to enrich the context in computer mediated communications.
        \item Built sentiment classifiers from LiveJournal posts with pre-trained word embedding as features.
    \end{itemize}

    \item \textbf{Augmented NTUSD}: Built a Chinese sentiment dictionary containing polarity information of words for use of research on sentiment analysis. Designed experiments to test the applicability of the dictionary.
\end{itemize}

% \faMapMarker~\textbf{National Taiwan University}, Graduate Research Assistant \hfill Sep. 2012 - June 2014 \par
% \textbf{Thesis}: Investigation on Optimal Mixing with Linkage Sets and Its Application \\
% \textbf{LTGA Improvement}: Analyzed on a theoretical basis how the linkage-tee genetic algorithm (LTGA) utilizes its linkage model and improved LTGA by applying a dynamic model throughout the optimizing process. \\
% \textbf{GA Benchmark Design}: Analyzed on a theoretical basis what problems are difficult for genetic algorithms (GA-hard) and designed a benchmark with controllable problem complexity for testing modern GAs.


% \section{ACHIEVEMENT \& AWARDS}
%
% \textbf{Fellowship}, University of California, Santa Cruz \hfill Jan. 2017 - Mar. 2017
%
% \textbf{Teaching Assistantship}, University of California, Santa Cruz \hfill Sep. 2016 - present
%
% \textbf{Travel Grant of Domestic Graduate Attending International Symposiums}, Ministry of Science and Technology, Taiwan \hfill Sep. 2014

% \textbf{Teaching Assistantship}, National Taiwan University \hfill Sep. 2013 - June 2014
% \begin{itemize}
%     \item Available to top 10\% of graduate students.
% \end{itemize}


\section{Teaching Experience}
\textbf{Teaching Assistant, University of California, Santa Cruz}  \hfill Sep. 2016 - present
    \begin{itemize}
        \item Assisted with course: Introduction to Programming in Python (CMPS 5p), Algorithms and Abstract Data Types (CMPS101), Adcanced Programming (CMPS 109), Comparitive Programming Languanges (CMPS 112)
        \item Conducted weekly lab sessions
        \item Provided individual and small group instruction during office hours
        \item Graded homework \& exams
    \end{itemize}
\textbf{Teaching Assistant, National Taiwan University} \hfill Sep. 2013 - June 2014
    \begin{itemize}
        \item Assisted with courses in Probability and Statistic, Algorithm, Genetic Algorithm
        \item Gave review lectures with designed problems
        \item Graded homework \& exams
    \end{itemize}

\section{ACHIEVEMENT \& AWARDS}

\textbf{Fellowship}, University of California, Santa Cruz \hfill Jan. 2017 - Mar. 2017

\textbf{Teaching Assistantship}, University of California, Santa Cruz \hfill Sep. 2016 - present

\textbf{Travel Grant of Domestic Graduate Attending International Symposiums}, Ministry of Science and Technology, Taiwan \hfill Sep. 2014

\textbf{Teaching Assistantship}, National Taiwan University \hfill Sep. 2013 - June 2014
\begin{itemize}
    \item Available to top 10\% of graduate students.
\end{itemize}


\section{Projects}
\href{https://github.com/ipod825/keraflow}{\textbf{Keraflow}}: Personal open source project implementing a deep learning library on top of \textbf{Theano} and \textbf{Tensorflow}. Redesigned the popular deep learning library Keras with the same utility but a simpler architecture aiming for easier development for package developers and clearer core dump for package users. \\
\href{https://github.com/ipod825/vim-netranger}{\textbf{vim-netranger}}: Personal open source project. A ranger-like system/cloud storage explorer plugin for Vim, bringing together the best of Vim, ranger, and rclone.  \\
\textbf{Recommendation System on Yelp Data} Course project for Machine Learning class (UCSC). Implemented (using only scipy \& numpy) a hybrid collaborative filtering model combining neighborhood model and factorization model, trained by batch gradient descent. \\
\textbf{Chinese OCR}: Course project for the OCR contest in Machine Learning class (NTU). Implemented algorithms including naieve Bayes, support vector machine, neural network.
\textbf{ChenLianYen}(Javascript): Course project implementing a web-based RPG game, where the user can control the role to explore the map and attack monsters on the map.


\section{Extracurricular}
Swimming Team of Medicine School, National Taiwan University \hfill 2012-2014
\begin{itemize}
    \item Finisher, 1.25 miles open water swimming
\end{itemize}

Taichung City Alumni Club, National Taiwan University \hfill 2008-2011
\begin{itemize}
    \item Deputy Director of Activity Group
    \item Convener of summer camp for freshman
\end{itemize}

\section{Skills}
\textbf{Programming Language}: C/C++, Java, Python, R, MATLAB, Javascript \\
\textbf{Machine Learning Libraries}: scipy, scikit-learn, Stanford CoreNLP, Keras, Theano, tensorflow \\
\textbf{General}: Git, neovim, pandas, matplotlib 
\textbf{Foreign Languages}: Chinese (native), Taiwanese (intermediate)

\end{document}
